\documentclass[modern]{aastex61}

% TODO
% ----

% Style guidelines
% ----------------
% - Use the Makefile; don't be typing ``pdflatex'' or some bullshit.
% - Wrap text to 80 character lines.
% - Line break between sentences to make the git diffs readable.
% - Use \, as a multiply operator.
% - Reserve () for function arguments; use [] or {} for outer shit.
% - Use \sectionname not Section, \figname not Figure, \documentname not Article

\include{gitstuff}
% Load common packages

\usepackage{amsmath}
\usepackage{amsfonts}
\usepackage{amssymb}
\usepackage{booktabs}

\usepackage{graphicx}
\usepackage{color}

\usepackage{hyperref}
\definecolor{niceblue}{rgb}{0.0, 0.4, 0.65}
\definecolor{linkcolor}{rgb}{0.02,0.35,0.55}
\definecolor{citecolor}{rgb}{0.4,0.4,0.4}
\hypersetup{colorlinks=true,linkcolor=linkcolor,citecolor=citecolor,
            filecolor=linkcolor,urlcolor=linkcolor}
\hypersetup{pageanchor=false}

\newcommand{\documentname}{\textsl{Article}}
\newcommand{\sectionname}{Section}
\newcommand{\figname}{Figure}
\newcommand{\eqname}{Equation}
\newcommand{\tblname}{Table}

% Packages / projects / programming
\newcommand{\package}[1]{\texttt{#1}}
\newcommand{\acronym}[1]{{\small{#1}}}
\newcommand{\github}{\package{GitHub}}
\newcommand{\python}{\package{Python}}

% Missions
\newcommand{\project}[1]{\textsl{#1}}

% For referee
\newcommand{\changes}[1]{{\color{red} #1}}

% Stats / probability
\newcommand{\given}{\,|\,}
\newcommand{\norm}{\mathcal{N}}

% Maths
\newcommand{\dd}{\mathrm{d}}
\newcommand{\transpose}[1]{{#1}^{\!\mathsf{T}}}
\newcommand{\inverse}[1]{{#1}^{-1}}
\newcommand{\argmin}{\operatornamewithlimits{argmin}}
\newcommand{\argmax}{\operatornamewithlimits{argmax}}
\newcommand{\mean}[1]{\left< #1 \right>}
\renewcommand{\vec}[1]{\bs{#1}}
\newcommand{\mat}[1]{\mathbf{#1}}

% Unit shortcuts
\newcommand{\msun}{\ensuremath{\mathrm{M}_\odot}}
\newcommand{\kms}{\ensuremath{\mathrm{km}~\mathrm{s}^{-1}}}
\newcommand{\au}{\ensuremath{\mathrm{au}}}
\newcommand{\pc}{\ensuremath{\mathrm{pc}}}
\newcommand{\kpc}{\ensuremath{\mathrm{kpc}}}
\newcommand{\kmskpc}{\ensuremath{\mathrm{km}~\mathrm{s}^{-1}~\mathrm{kpc}^{-1}}}

% Misc.
\newcommand{\bs}[1]{\boldsymbol{#1}}

% Astronomy
\newcommand{\DM}{{\rm DM}}
\newcommand{\feh}{\ensuremath{{[{\rm Fe}/{\rm H}]}}}
\newcommand{\df}{\acronym{DF}}

% TO DO
\newcommand{\todo}[1]{{\color{red} TODO: #1}}

\graphicspath{{figures/}}

% Extra packages
\usepackage{microtype}
\usepackage{multirow}

% Change the citation color or link color here:
% \definecolor{linkcolor}{rgb}{0.02,0.35,0.55}
% \definecolor{citecolor}{rgb}{0.4,0.4,0.4}

\newcommand{\Rcor}{\ensuremath{R_{\rm co}}}

\begin{document}\sloppy\sloppypar\raggedbottom\frenchspacing % trust me

\title{Chaos in the inner halo from the Galactic bar}

\author[0000-0003-0872-7098]{Adrian~M.~Price-Whelan}
\affiliation{Department of Astrophysical Sciences,
             Princeton University, Princeton, NJ 08544, USA}

% \author[0000-0002-5151-0006]{David~N.~Spergel}
% \affiliation{Flatiron Institute,
%              Simons Foundation,
%              162 Fifth Avenue,
%              New York, NY 10010, USA}
% \affiliation{Department of Astrophysical Sciences,
%              Princeton University, Princeton, NJ 08544, USA}

\correspondingauthor{Adrian M. Price-Whelan}
\email{adrn@astro.princeton.edu}

\begin{abstract}
% Context
The Milky Way's triaxial bulge/bar is a significant component of the mass
distribution in the inner Galaxy, accounting for $\approx$1/6 of the visible
mass of the disk.
Bars alter the orbit structure of a galaxy by creating resonances and chaotic
orbits.
Chaotic orbits cause disrupted kinematic substructures---e.g., stellar streams
or open clusters---to mix more rapidly than around purely regular orbits.
The Galactic bar will therefore have a significant impact on the amount of
kinematic substructure in the central Galaxy and inner stellar halo.
% Aims
In this work, we study the types of orbits in the inner disk and halo of a Milky
Way-like mass model that includes a triaxial bar.
We then measure the mixing rate of cold substructures around these orbits.
% Methods
We first map the orbit structure of our Milky Way model by computing fundamental
frequencies and frequency diffusion rates in a rotating frame for grids of
orbits with constant Jacobi energy.
We vary the pattern speed (and therefore corotation radius and bar length)
from 30--60~\kmskpc, a range of pattern speeds that brackets the plausible
pattern speeds for the Galactic bar.
We then perform the same tests using orbits sampled from stellar halo-like
distribution functions with different anisotropy parameters to see how the
number of strongly chaotic orbits depends on the bar mass and pattern speed.
% Results
% Conclusions
\end{abstract}

\keywords{
    whatever: whatever
    ---
}

\section{Introduction} \label{sec:intro}

Milky Way has a bar

Bar known to create chaotic orbits

Bar has been shown to affect streams

What fraction of inner halo orbits are strongly chaotic?
Does this erase substructure?

\section{Methods} \label{sec:methods}

\subsection{Mass model of the Milky Way} \label{sec:potential}

The total gravitational field of the Milky Way is time-dependent and rich with
substructure on a large range of scales.
[The different components of the Galaxy have formed and evolved in different
ways]
[Still, we've learned a lot using simple static models.]
Here, we construct a simple mass model that contains smooth representations of
the dominant components of the Milky Way's mass distribution: the stellar and
gas disks, stellar bar, and dark matter halo.
[No external perturbations, e.g., Sgr. No spiral structure, evolving disk,]

Our Milky Way mass model accounts for the stellar and gas disks, dark matter
halo, and bar/bulge.
\todo{disk, halo}

For the Galactic bar component, we use a triaxial, Gaussian density
distribution (model G2 in \citealt{Dwek:1995})
\begin{eqnarray}
    \rho(\bs{x}) &=& \rho_0 \, \exp\left(-r_1^2 / 2 \right)\\
    r_1 &=& \left[\left\{(x/x_0)^2 + (y/y_0)^2\right\}^2 + (z/z_0)^4 \right]^{1/4} \quad .
\end{eqnarray}
This model has been shown to match the projected surface-brightness of the bulge
region (\citealt{Dwek:1995}) and velocity and velocity dispersion measurements
of the bulge from the \acronym{BRAVA} survey (\citealt{Kunder:2012,Wang:2012}).
Following the original model (\citealt{Dwek:1995}), given a pattern speed, we
truncate the bar model at the corotation radius, \Rcor, using a function of
cylindrical radius $R = \sqrt{x^2 + y^2}$
\begin{equation}
    f(R) =
    \begin{cases}
        1 & R < \Rcor \\
        \exp\left(-R^2/R_0^2\right) & R \geq \Rcor
    \end{cases}\label{eq:truncfunc}
\end{equation}
with $R_0 = 0.5~\kpc$.
This density distribution does not have an analytic potential expression;
we instead use a basis function expansion to represent the density and
potential, and compute forces from this component (as is done in, e.g.,
\citealt{Wang:2012}).
We use the self-consistent field (SCF) expansion formalism
(\citealt{Hernquist:1992}) to compute the expansion coefficients and truncate
the expansion at $n=6$, $l=6$.
We find that with this expansion, the force errors in the volume $0.1~\kpc < r <
30~\kpc$ are typically $<1\%$, where $r$ is the Galactocentric radius.
\todo{double check this}

In the experiments below, we consider a grid of pattern speeds between $30 \leq
\Omega_{\rm b} \leq 60~\kmskpc$.
We compute a different set of expansion coefficients for each bar pattern speed:
the pattern speed sets the corotation and therefore truncation radius
(\eqname~\ref{eq:truncfunc}) of the bar model.

\subsection{Orbit integration} \label{sec:orbit-int}

We use an 8th-order Runge-Kutta scheme to integrate orbits in the above mass
model (the Dormand-Prince method; \citealt{Dormand:1980})
In detail, we use a \python\ wrapper of a \texttt{C} implementation
(\citealt{Hairer:1993}) that is implemented in the \package{Gala} package
(\citealt{Gala}).
For all orbits we adjust the timestep until the Jacobi energy is conserved to at
least $|\Delta E/E_0| \leq 10^{-8}$ by the end of integration, but most regular
orbits conserve energy at machine precision ($|\Delta E/E_0| \leq 10^{-15}$).

\todo{Timestep stuff...}
Unless otherwise specified the integration timesteps are
chosen so that there are 512 steps per strongest orbital period component, but
the integrator uses adaptive stepping between each main step in order to satisfy
a specified tolerance (we set the absolute tolerance to $\approx$100 times
machine precision, $\texttt{atol} = 10^{-13}$).

\subsection{Frequency estimation and diffusion rate} \label{sec:freqdiff}

TODO

\section{Results} \label{sec:results}

\subsection{Orbit structure: grids at constant Jacobi energy}

TODO: frequency maps and frequency diffusion rates

\subsection{Orbit structure: sampling from embedded DFs}

Vary anisotropy of DF

How many strongly chaotic orbits as a function of bar mass, pattern speed?

\subsection{Mixing of substrcture}

Timescale for a globular cluster-like clump of debris to phase-mix?

\acknowledgements

It is a pleasure to thank

\software{
The code used in this project is available from
\url{https://github.com/adrn/GaiaPairsFollowup} under the MIT open-source
software license.
This research utilized the following open-source \python\ packages:
    \package{Astropy} (\citealt{Astropy-Collaboration:2013}),
    % \package{astroquery} (\citealt{Ginsburg:2016}),
    % \package{ccdproc} (\citealt{Craig:2015}),
    % \package{celerite} (\citealt{Foreman-Mackey:2017}),
    % \package{corner} (\citealt{Foreman-Mackey:2016}),
    % \package{emcee} (\citealt{Foreman-Mackey:2013ascl}),
    \package{IPython} (\citealt{Perez:2007}),
    \package{matplotlib} (\citealt{Hunter:2007}),
    \package{numpy} (\citealt{Van-der-Walt:2011}),
    \package{scipy} (\url{https://www.scipy.org/}),
    \todo{schwimmbad, gala, superfreq}
    % \package{sqlalchemy} (\url{https://www.sqlalchemy.org/}).
% This work additionally used the Gaia science archive
% (\url{https://gea.esac.esa.int/archive/}), and the SIMBAD database
% (\citealt{Wenger:2000}).
}

% \facility{MDM: Hiltner (Modspec)}

% \appendix

% \section{Some extra stuff} % \label{appdx:}

\clearpage
\bibliographystyle{aasjournal}
\bibliography{refs}

\end{document}
