\documentclass[modern]{aastex61}

% TODO
% ----

% Style guidelines
% ----------------
% - Use the Makefile; don't be typing ``pdflatex'' or some bullshit.
% - Wrap text to 80 character lines.
% - Line break between sentences to make the git diffs readable.
% - Use \, as a multiply operator.
% - Reserve () for function arguments; use [] or {} for outer shit.
% - Use \sectionname not Section, \figname not Figure, \documentname not Article

\include{gitstuff}
\include{preamble}
\graphicspath{{figures/}}

% Extra packages
\usepackage{microtype}
\usepackage{multirow}

% Change the citation color or link color here:
% \definecolor{linkcolor}{rgb}{0.02,0.35,0.55}
% \definecolor{citecolor}{rgb}{0.4,0.4,0.4}

\begin{document}\sloppy\sloppypar\raggedbottom\frenchspacing % trust me

\title{Chaos in the inner halo from the Galactic bar}

\author[0000-0003-0872-7098]{Adrian~M.~Price-Whelan}
\affiliation{Department of Astrophysical Sciences,
             Princeton University, Princeton, NJ 08544, USA}

% \author[0000-0002-5151-0006]{David~N.~Spergel}
% \affiliation{Flatiron Institute,
%              Simons Foundation,
%              162 Fifth Avenue,
%              New York, NY 10010, USA}
% \affiliation{Department of Astrophysical Sciences,
%              Princeton University, Princeton, NJ 08544, USA}

\correspondingauthor{Adrian M. Price-Whelan}
\email{adrn@astro.princeton.edu}

\begin{abstract}
% Context
% Aims
% Methods
% Results
% Conclusions
\end{abstract}

\keywords{
    whatever: whatever
    ---
}

\section{Introduction} \label{sec:intro}

Milky Way has a bar

Bar known to create chaotic orbits

Bar has been shown to affect streams

What fraction of inner halo orbits are strongly chaotic?
Does this erase substructure?

\section{Methods} \label{sec:methods}

\subsection{Gravitational potential of the Milky Way} \label{sec:potential}

Use bar model G2 from Dwek 1995 with truncation at corotation radius,
explain $f(\bs{r})$

Basis function expansion, similar to Wang 2012, taken to higher order, but we
compute the expansion for different pattern speeds -> corotation radii.

Use a grid of pattern speeds from 30--60~$\kmskpc$.

\subsection{Orbit integration} \label{sec:orbit-int}

TODO

\subsection{Chaos indicator and frequency diffusion rate} \label{sec:freqdiff}

TODO

\section{Results} \label{sec:results}

\subsection{Orbit structure: grids at constant Jacobi energy}

TODO: frequency maps and frequency diffusion rates

\subsection{Orbit structure: sampling from embedded DFs}

Vary anisotropy of DF

How many strongly chaotic orbits as a function of bar mass, pattern speed?

\subsection{Mixing of substrcture}

Timescale for a globular cluster-like clump of debris to phase-mix?

\acknowledgements

It is a pleasure to thank

\software{
The code used in this project is available from
\url{https://github.com/adrn/GaiaPairsFollowup} under the MIT open-source
software license.
This research utilized the following open-source \python\ packages:
    \package{Astropy} (\citealt{Astropy-Collaboration:2013}),
    % \package{astroquery} (\citealt{Ginsburg:2016}),
    % \package{ccdproc} (\citealt{Craig:2015}),
    % \package{celerite} (\citealt{Foreman-Mackey:2017}),
    % \package{corner} (\citealt{Foreman-Mackey:2016}),
    % \package{emcee} (\citealt{Foreman-Mackey:2013ascl}),
    \package{IPython} (\citealt{Perez:2007}),
    \package{matplotlib} (\citealt{Hunter:2007}),
    \package{numpy} (\citealt{Van-der-Walt:2011}),
    \package{scipy} (\url{https://www.scipy.org/}),
    \todo{schwimmbad, gala, superfreq}
    % \package{sqlalchemy} (\url{https://www.sqlalchemy.org/}).
% This work additionally used the Gaia science archive
% (\url{https://gea.esac.esa.int/archive/}), and the SIMBAD database
% (\citealt{Wenger:2000}).
}

% \facility{MDM: Hiltner (Modspec)}

% \appendix

% \section{Some extra stuff} % \label{appdx:}

\clearpage
\bibliographystyle{aasjournal}
\bibliography{refs}

\end{document}
